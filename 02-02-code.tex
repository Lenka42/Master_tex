\chapter{Code discussion}
TB calculations, which results are presented in this work, are performed using package I wrote in Python programming language. Input should be also written, but as far as Python designed to be highly readable, even person who never programmed in Python before, can easily do this. 

In this chapter I'll discuss an basic input example and what's going on when one runs it for calculating graphene band structure (Results of this input are presented in chapter \ref{subsec:graphene_result})

\begin{python}
from atom.model import Atom
from system.model import System
from plotter.plotter import Plotter

from numpy import array, sqrt, pi

d = 1.0
system = System([d / 2. * array([3., sqrt(3), 0.]),
                 - d / 2. * array([3, -sqrt(3), 0.])],
                mode='with_overlap', name='grapene_sp')
\end{python}
 
At the beginning of the input file I do all the imports. First three are classes from the TightBinding package and the last is necessary functions from NumPy library. NumPy is a high-level package for efficient manipulating multi-dimensional arrays in Python. 

The next line is defining a constant, which here means distance of $C-C$ bond in graphene.

At the beginning of logical part I define \verb!System! object. The first argument for system object initiating is list of lattice vectors. Length of this list may be from $0$ to $3$. Next argument is mode, it can be either \verb!"standard"!, \verb!"with_overlap"! when overlap matrix is not identity matrix or \verb!"with_vectors"! when one also wants eigenstates to be calculated. Name of the system is used to create directory with outputs.

\begin{python}
system.atoms = [Atom('C', array([d, 0., 0.])),
                Atom('C', array([2 * d, 0., 0.])), ]
\end{python}

Next I define list of atoms in the unit cell. Each atom has a name and a position.

\begin{python}
for i in xrange(len(system.atoms)):
    system.atoms[i].orbitals = ['s', 'px', 'py', 'pz', ]
\end{python}

Here I assign to each atom in a system a set of orbitals. I do this in a loop, but one can also assign different sets of orbitals for every atom one by one.

\begin{python}
system.k_points = [array([0., 0., 0.]),
                   array([2 * pi / 3 / d, 2 * pi / 3 / sqrt(3) / d, 0]),
                   array([2 * pi / 3 / d, 0, 0]),
                   array([0., 0., 0.])]
system.make_k_mesh(100)
\end{python}

Thereafter one should define mesh of wave vector points. One can do this manually setting parameter \verb!system.k_mesh! or as above defining \verb!system.k_points! and calling method \verb!make_k_mesh!. In the snippet above I define a path of $\Gamma$, $K$, $M$ and again $\Gamma$ points and call the function \verb!make_k_mesh! to create a mesh with approximately $100$ points following the path defined in the previous line.

\begin{python}
system.parameters = {
    'C': {
        'es': -8.7,
        'ep': 0.0,
    },
    'CC': {
        'Vsss': - 6.7,
        'Vsps': 5.5,
        'Vpps': 5.1,
        'Vppp': -3.1,
    }
}
system.s_parameters = {
    'C': {
        'es': 1.,
        'ep': 1.,
    },
    'CC': {
        'Vsss': 0.2,
        'Vsps': - 0.1,
        'Vpps': - 0.15,
        'Vppp': 0.12,
    }
}
\end{python}

Subsequently I define an intuitive set of parameters for my system.

\begin{python}
system.just_do_main_magic()
plotter = Plotter(system.name)
plotter.plot_energy_bands_from_file()
\end{python}

Finally, I call the function \verb!just_do_main_magic! which does the main logic of the TB calculations. After I plot the electronic structure (this step is highly optional, as far as the results are stored in the files).

Here I will explain, what is hidden under the function \verb!just_do_main_magic!. First helper function, which searches for nearest neighbors of each atom is called. Then for every wave vector Hamiltonian (and overlap if needed) matrix is determined. After eigenvalue problem is solved. I used functions from SciPy package for diagonalization of Hermitian matrices.

Also, after the main band structure calculation one may calculate density of states (DOS) or local density of states (LDOS) for any atom in the system or type of orbital.
\begin{python}
doser = DOSCalculator(system.dim, system.name, n)
doser.f()
\end{python}
Here first line is creation of helper object for DOS calculation, first argument is periodicity of the system, and \verb!n! is name of the system and the last is number of points in energy domain, for which one wants to calculate density of states. The next line is call of the function, which calculates DOS, writes results to the file and plots it.

One can calculate DOS in energy domain as follows: 
\begin{equation}
	\rho(E) = \int\limits_{BZ} d\vec{k} \sum\limits_i \delta(\epsilon_i(\vec{k}) - E),
\end{equation}
where integration is over Brillouin zone and summation goes over energy bands. But results of TB calculations, $\epsilon_i(\vec{k})$ is represented as a set of values for discrete set of k-vector points. So in my calculations I create an interpolated in whole Brillouin zone continuous function. And replace delta functional with narrow Gaussian function:
\begin{equation}
	\rho(E) = \int\limits_{BZ} d\vec{k} \sum\limits_i \frac{1}{a\sqrt{\pi}} \exp\left(\frac{(\epsilon_i(\vec{k}) - E)^2}{a^2}\right)
\end{equation}
I've found, that $a = \frac{\max(\epsilon(\vec{k})) - \min(\epsilon(\vec{k}))}{n}$ works quit good.

