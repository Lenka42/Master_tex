\chapter{Tight Binding method}
\label{ch:theory}
\section{Introduction to the tight-binding model}
Tight-binding method is developed to describe band structure of various materials, where electrons are localized at the atomic positions (tight-bonded to the atoms). In TB model simple effective hamiltonian is used in contrast to first-principle calculations. 

Hear I provide short sketch of secular equation derivation.

The most essential approximation in TB model is the so-called two-center approximation, where Hamiltonian is approximated by the atomic Hamiltonian centered on the atomic positions in the unit cell $\vec{R}$ and only two-centered integrals are encountered.

Crystalline solids have translational symmetry along the directions of lattice vactors $a_i$. In further chapters I investigate both regular bulk materials, which have translational symmetry in three dimensions and nanomaterials, which have reduced symmetry. Translational symmetry enforces Bloch's thorem to be satisfied by solids' wave functions:
\begin{equation} \label{eq:blochs_theorem}
	\hat{T}_i \psi(\vec{k},\vec{r}) = \me^{i \vec{k} \cdot \vec{a_i}} \psi(\vec{k}\vec{r}),
\end{equation}
where $\hat{T}_i$ is the translational operator along the lattice vector $\vec{a_i}$ and $\vec{k}$ is the Bloch wave vector \cite{kittel}.

We construct the Bloch wave function of the crystal as a linear combination of the local Wannier functions, which are approximated by the eigenfunctions of the atomic Hamiltonian, the atomic orbitals $\phi_{nlm, s}(\vec{r} - \vec{t_i} - \vec{R_j})$, where $\vec{t_i}$ is the position of atom $i$ in the unit cell at cite $\vec{R_j}$, $s$ is the spin and $n$, $l$, $m$ are the principal, angular-momentum and magnetic quantum numbers, respectively. The Bloch wave function looks as follows:
\begin{equation} \label{eq:wave_fun}
	\psi_{j}(\vec{k},\vec{r}) = \frac{1}{\sqrt{N}} \sum_{\vec{t_i}, nlms} \me^{i\vec{k}(\vec{R_j} + \vec{t_i})} \phi_{nlm, s} (\vec{r} - \vec{t_i} - \vec{R_j}),
\end{equation}
where N is the number of atomic wave functions in the unit cell. It's easy to see that constructed in such a way wave function satisfies Bloch's theorem (eq. \ref{eq:blochs_theorem}).

Hamiltonian's eigenfunctions $\Psi_j(\vec{k}, \vec{r})$ are defined as a linear combination of Bloch functions:
\begin{equation} \label{eq:wave_function_expansion}
	\Psi_j(\vec{k}, \vec{r}) = \sum_{j'=1}^{n} C_{jj'}(\vec{k}) \psi_j(\vec{k}, \vec{r})
\end{equation}

Energies of the system characterized by Hamiltonian $\hat{H}$ are given by
\begin{equation}
	E_i(\vec{k}) = \frac{\langle \Psi_i | \hat{H} | \Psi_i \rangle}{\langle \Psi_i | \Psi_i \rangle} = \frac{\int \Psi^*_j(\vec{k}, \vec{r}) \hat{H} \Psi_j(\vec{k}, \vec{r}) dr}{\int \Psi^*_j(\vec{k}, \vec{r}) \Psi_j(\vec{k}, \vec{r}) dr}
\end{equation}
Expantion of $\Psi_j(\vec{k}, \vec{r})$ according to eq. \ref{eq:wave_function_expansion} leads to
\begin{equation}
	E_i(\vec{k}) = \frac{\sum\limits_{j,j'=1}^{n} C^*_{ij} C_{ij'} \langle \psi_j | \hat{H} | \psi_{j'} \rangle}{\sum\limits_{j,j'=1}^{n} C^*_{ij} C_{ij'} \langle \psi_j | \psi_{j'} \rangle} = \frac{\sum\limits_{j,j'=1}^{n} H_{jj'}(\vec{k}) C^*_{ij} C_{ij'}}{\sum\limits_{j,j'=1}^{n} S_{jj'} C^*_{ij} C_{ij'}},
\end{equation} 
where $H_{jj'}(\vec{k})$ and $S_{jj'}(\vec{k})$ are hamiltonian (transfer) and overlap matrices elements, respectively, and are defined by
\begin{equation}
	H_{jj'}(\vec{k}) = \langle \psi_j | \hat{H} | \psi_{j'} \rangle
\end{equation}
\begin{equation}
	S_{jj'}(\vec{k}) = \langle \psi_j | \psi_{j'} \rangle
\end{equation}
For a given $\vec{k}$ value, coefficient $C^*_{ij}(\vec{k})$ is optimized to minimize $E_i(\vec{k})$:
\begin{equation}
	\frac{\partial E_i (\vec{k})}{\partial C^*_{ij}(\vec{k})} = \frac{\sum\limits_{j'=1}^{n} H_{jj'}(\vec{k}) C_{ij'}(\vec{k})}{\sum\limits_{j,j'=1}^{n} S_{jj'} C^*_{ij} C_{ij'}} - \frac{\sum\limits_{j,j'=1}^{n} H_{jj'} C^*_{ij} C_{ij'}}{[\sum\limits_{j,j'=1}^{n} S_{jj'} C^*_{ij} C_{ij'}]^2} \sum\limits_{j'=1}^{n} S_{jj'}(\vec{k}) C_{ij'}(\vec{k}) = 0.
\end{equation}
This can be simplified as 
\begin{equation}
	\sum_{j'=1}^{n} H_{jj'}(\vec{k}) C_{ij'}(\vec{k}) - E_i(\vec{k}) \sum_{j'=1}^{n} S_{jj'}(\vec{k}) C_{ij'}(\vec{k}) = 0
\end{equation}

Usually integration results are taken as parameters, which have to be fitted to reproduce certain solid's properties or the band structure calculated by first-principle approach. Matrix Elements in \ref{eq:h_matrix} and \ref{eq:s_matrix} where $\vec{R} = \vec{R'}$ are called on-site elements, and for $\vec{R} \neq \vec{R'}$ they are called hopping and overlap parameters respectively. 

In general case, atomic orbitals centered on different sites are not orthogonal and the corresponding overlap parameters have small but finite values.But often orthogonal basis of orbitals is used, so that overlap matrix becomes identity matrix and energies are eigenvalues of the Hamiltonian matrix. In the following sections I will discuss both approaches. 

The next approximation is to consider a finite as small as necessary number of orbitals per atom. The number of solutions of the secular equation in \ref{eq:secular} is equal to dimension of the Hamiltonian matrix,
\begin{equation}
dim = g \times \sum_j n_j
\end{equation}
Here $g$ is spin factor ($1$ or $2$), summation goes through all atoms in unit cell and $n_j$ is number of included orbitals for atom $j$.

In the final nearest neighbors approximation (NNA) only the nearest neighbors of a chosen atom are taken into account in the Hamiltonian and overlap matrix elements Eqs. (\ref{eq:h_matrix}) and (\ref{eq:s_matrix}).
\section{Spin-orbit interaction }
