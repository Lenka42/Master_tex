\chapter{Graphene}
\section{Very brief history of graphene}
Graphene is the most well-known 2D-material in the world.  Explosion of great interest to this material happened after Andre Geim and Konstantin Novoselov in 2010 received Nobel Prize in Physics "for groundbreaking experiments regarding the two-dimensional material graphene" \cite{geim}. 

But story of graphene started long years ago. Already in 1859 D.C. Brodie discovers the highly lamellar structure of thermally reduced graphite oxide \cite{brodie}. At the beginning of XXth century crystal structure of graphite was solved with diffraction method \cite{debije, bernal} and properties of graphite oxide paper was studied \cite{haenni}.

The theory of graphene was first studied by P. R. Wallace in 1947 as a first step to describing electronic properties of 3D graphite \cite{wallace}. The massless Dirac equation was first formulated for graphene by David P. DiVincenzo and Eugene J. Mele in 1984 \cite{divincenzo}. Semenoff emphasized the occurrence in a magnetic field of an electronic Landau level precisely at the Dirac point. This level is responsible for the anomalous integer quantum Hall effect.

At the ed of XXth century single layers of carbon atoms where epitaxially grown on other bulk materials\cite{epitaxial}. However strong interaction between graphene grown in such way and substrate significantly affects electronic structure and properties of graphene. In 2004 Andre Geim and Konstantin Novoselov published their work on the scotch tape method witch allowed to produce single layers of graphene weakly bounded to substrate on relatively large scale\cite{geim-science}. This paper initiated a global explosion in graphene research.

\section{Geometry of graphene}
Graphene is one-atom-sick sheet of carbon atoms arranged in hexagonal as shown on fig. \ref{fig:graphene_lattice}. The lattice vectors can be chosen as 
\begin{equation}
	\vec{a_1} = \frac{a}{2} (3, \sqrt{3}), \qquad \vec{a_2} = \frac{a}{2} (3, - \sqrt{3}),
\end{equation}
where $a \approx 1.42 \AA$ is $C-C$ bond length. There are two carbon atoms per unit cell, let's name them $A$ and $B$.The three nearest-neighbor vectors in real space are given by 
\begin{equation}
	\vec{\delta_1} = a (1, 0), \quad \vec{\delta_2} = \frac{a}{2} (-1, \sqrt{3}), \quad \vec{\delta_3} = \frac{a}{2} (-1, -\sqrt{3})
\end{equation}
and point from atom of one kind to another, e.g. from $A$ to $B$ and vice versa.
The reciprocal space lattice are given by
\begin{equation}
	\vec{b_1} = \frac{2 \pi}{3 a}(1, \sqrt{3}), \qquad \vec{b_2} = \frac{2 \pi}{3 a} (1, -\sqrt{3})
\end{equation}
Of particular interest for physics of graphene are two points in reciprocal space at the corners of Brillouin zone (BZ):
\begin{equation}
	K = (\frac{2 \pi}{3 a}, \frac{2 \pi}{3 \sqrt{3}}), \qquad K' = (\frac{2 \pi}{3 a}, -\frac{2 \pi}{3 \sqrt{3}})
\end{equation}

\begin{figure}[ht]
\begin{center}
  \includegraphics[width=0.55\linewidth]{img/graphene_lattice}
  \caption{Left: lattice structure of graphene. Right: corresponding Brillouin zone \label{fig:graphene_lattice}}
\end{center}
\end{figure}

\section{Calculation of $\pi$ bands}
The easiest model one can use to describe electronic properties of graphene is TB model considering only $p_z$ orbitals for each carbon atom. All this orbitals are perpendicular to graphene plane and parallel to each other. They form $\pi$ bands. Band structure of graphene around the Fermi level is determined by the $\pi$ bands. 

Here I will take into account only interaction between nearest neighbors. One I assume, that orbitals at different atoms are orthogonal to each other.

So to determine graphene band structure one should solve eigenvalue problem, where Hamiltonian matrix dimension is $2 \times 2$. The energy bands derived from this Hamiltonian have the form \cite{wallace}
\begin{equation} \label{eq:dispersion}
	E(\vec{k}) = \pm V_{pp\pi}\sqrt{3 + 2 \cos\left(\sqrt{3} k_y a\right) + 4 \cos\left(\frac{\sqrt{3}}{2} k_y a\right) \cos\left(\frac{3}{2} k_x a\right)} 
\end{equation}
Such model gives two energy bands, which are symmetric.

Plot of graphene electronic structure obtained with my code and analytical function is shown on fig. \ref{fig:graphene_pz}
\begin{figure}[ht]
\begin{center}
  \includegraphics[width=0.6\linewidth]{img/graphene_pz}
  \caption{Calculated and analytical band structure of graphene ($p_z$ orbitals)) \label{fig:graphene_pz}}
\end{center}
\end{figure}

\section{Non-zero overlap}
\label{subsec:graphene_result}
Approach discussed in previous chapter gives the result spectrum which is symmetric around zero energy (easy to see in eq. \ref{eq:dispersion}). The electron-hole symmetry could be broken using two approaches in TB model. First is including second-nearest neighbors (hopping between atoms of the same type in neighboring cells). And second is including non-trivial overlap matrix. Here I discuss the second method.
In my calculations I took $s$, $p_x$, $p_y$ and $p_z$ for each carbon atom. Hopping and overlap parameters \cite{basics} are placed at the table \ref{tab:graphene_overlap_parameters}. Resulting band structure is presented on fig. \ref{fig:graphene_sp}. The agreement with results of first-principles calculations \cite{boukhvalov} (fig. \ref{fig:graphene_dft}) is quite good.

\begin{table}[h]
\begin{center}
\begin{tabular}{|c |c |c| c|}
\hline
Parameter&Value, [eV]&Parameter&Parameter\\ \hline
$\epsilon_s$ &$- 8.7$ & &\\ \hline
$\epsilon_p$&$0$ & &\\ \hline
$V_{ss\sigma}$&$- 6.7$ & $S_{ss\sigma}$&$0.20$\\ \hline
$V_{sp\sigma} $&$5.5$ & $S_{sp\sigma}$&$0.10$\\  \hline
$V_{pp\sigma}$&$5.1$ & $S_{pp\sigma}$&$- 0.15$\\ \hline
$V_{pp\pi}$&$- 3.1$ & $S_{pp\pi}$&$0.12$ \\ \hline  
\end{tabular}
\end{center}
\caption{Graphene TB hopping and overlap parameters \cite{basics}}
\label{tab:graphene_overlap_parameters}
\end{table}

\begin{figure}[hb] 
\begin{center}
  \includegraphics[width=0.6\linewidth]{img/graphene_sp}
  \caption{Calculated Band structure of graphene ($s$, $p$ orbitals). Red lines are $\pi$ bands, black lines are $\sigma$ bands, gray lines are $\pi$ bands calculated for orthogonal $p_z$ orbitals. \label{fig:graphene_sp}}
\end{center}
\end{figure}

\begin{figure}[h] 
\begin{center}
  \includegraphics[width=0.6\linewidth]{img/graphene_dft}
  \caption{Graphene band structure (DFT calculations). Solid red lines are $\sigma$ bands and dotted blue lines are $\pi$ bands. \cite{boukhvalov} \label{fig:graphene_dft}}
\end{center}
\end{figure}

\section{Spin-orbit coupling in graphene}
I included to the Hamiltonian SOC term, as described in chapter \ref{sec:soc_theory}. First I considered $\{ s, p_x, p_y, p_z \}$ set of orbitals. The SOC couples $p_x$ and $p_y$ orbitals. But SOC does not influence $\pi$ bands, so no splitting at the Dirac points appears $E_g = 0 eV$.

It was proven, that to the spin-orbit splitting at $K$ point in graphene one has to consider $d$
orbitals \cite{slonczewski}. Calculated graphene band structure for $\{p_z, d_{xy}, d_{yz}, d_{xz}, d_{x^2 - y^2}, d_{z^2} \}$ with set of parameters as in tab. \ref{tab:graphene_soc} is shown on fig. \ref{fig:graphene_soc}. For SOC parameter taken $\lambda=0.001$ intrinsic energy gap at $K$ point is $E_g \approx 43 \mu eV$.
\begin{table}[h]
 \begin{center}
  \begin{tabular}{|c|c|}
  \hline
    Parameter&Value, [eV]\\ \hline
    $\epsilon_p$ & $1.2057$ \\ \hline
    $\epsilon_d$ & $24.1657$ \\ \hline
    $V_{pp \pi}$ & $-3.2600$  \\ \hline
    $V_{pd \pi}$ & $2.4000$\\ \hline
    $V_{dd \pi}$ & $3.6000$ \\ \hline
    $V_{dd \delta}$ &  $-7.4000$\\ \hline
  \end{tabular}
 \end{center}
  \caption{Interaction parameters for diamond. \cite{graphene_soc_parameters}}
\end{table}
