\chapter{Introduction}
\label{ch:introduction}
Tight-binding (TB) method is an approach to the calculation of energy band structure of various materials, where electrons are tightly bound to the atom to which they belongs. In TB model simple effective Hamiltonian is used in contrast to first-principle calculations. And an approximate set of wave functions based on the superposition of wave functions for isolated atoms located at atomic sites is used. All these make TB approach a great tool for investigating quite big systems and getting quantitative results.

TB method was developed by F. Bloch \cite{bloch} in 1928. Bloch considered only the $s$ atomic orbital. In 1934 H. Jones, N. F. Mott and H. W. B. Skinner \cite{mott} considered wave functions constructed by solving a secular problem between $s$, $p_x$, $p_y$ and $p_z$ Bloch functions.

I developed a numerical package for Tight Binding calculations in Python programming language. With the help of this code one can investigate both regular bulk materials, which have translational symmetry in three dimensions and nanomaterials, which have reduced symmetry: $2$D -- nanosheets, surfaces, $1$D -- nanoribbons, nanorods, $0$D -- nanoparticles, molecules.

As an example of 3D system I provided results of band structure calculations for diamond. Furthermore I investigated graphene using different approaches and studied spin-orbit coupling effects. And finally I analyzed effects connected with periodicity of the system in graphene nanoribbons.

Package code and example inputs are placed in open repository \url{https://github.com/Lenka42/TightBinding}.