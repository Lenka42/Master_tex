\documentclass{my_pracamgr}

\usepackage{lmodern}
\usepackage{upgreek}
\usepackage[utf8]{inputenc}
\usepackage{graphicx}
\usepackage{listings}
\usepackage{epstopdf}
\usepackage{hyperref}
\usepackage{color}
\usepackage{footnote}
\usepackage{listings}
\usepackage{textcomp}
\usepackage[polish,english]{babel}
\usepackage[T1]{fontenc}
\usepackage{subcaption}
\usepackage{verbatim}
\usepackage{float}
\usepackage{tabularx}
\usepackage[vlines]{tabularht}

\newcommand{\me}{\mathrm{e}}
\DeclareFixedFont{\ttb}{T1}{txtt}{bx}{n}{9} % for bold
\DeclareFixedFont{\ttm}{T1}{txtt}{m}{n}{9}  % for normal
% Defining colors
\usepackage{color}
\definecolor{deepblue}{rgb}{0,0,0.5}
\definecolor{deepred}{rgb}{0.6,0,0}
\definecolor{deepgreen}{rgb}{0,0.5,0}
\newcommand\pythonstyle{\lstset{
  language=Python,
  backgroundcolor=\color{white}, %%%%%%%
  basicstyle=\ttm,
  otherkeywords={self},            
  keywordstyle=\ttb\color{deepblue},
  emph={MyClass,__init__},          
  emphstyle=\ttb\color{deepred},    
  stringstyle=\color{deepgreen},
  commentstyle=\color{red},  %%%%%%%%
  frame=tb,                         
  showstringspaces=false            
}}
\lstnewenvironment{python}[1][]
{
\pythonstyle
\lstset{#1}
}
{}
\author{Elena Shylko}

\nralbumu{320957}

\title{Tight Binding method for electronic structure of two-dimensional nanostructures}

\tytulang{Metoda ciasnego wiązania dla struktury elektronowej nanostruktur dwuwymiarowych}

\kierunek{Mathematical and computer modeling of physical processes}

\opiekun{Prof. Jacek A. Majewski\\
  Chair of Condensed Matter Physics\\
  }

\date{June 2016}

\dziedzina{ 
13.2 Physics\\ 
}

\def\category#1#2#3{%
    \begingroup
        \let\and\relax
            #1 [\textbf{#2}]%
                \if!#3!\else : #3\fi
    \endgroup
}
\klasyfikacja{
\category{71.15.Ap}{Methods of electronic structure calculations}{Basis sets (LCAO, plane-wave, APW, etc.) and related methodology (scattering methods, ASA, linearized methods, etc.)} \\
\category{71.15.Dx}{Methods of electronic structure calculations}{Computational methodology (Brillouin zone sampling, iterative diagonalization, pseudopotential construction).} \\
\category{73.22.Pr}{Electronic structure of nanoscale materials and related systems}{Electronic structure of graphene.}  }

\keywords{Tight binding, modelling, electronic sructure, low-dimensional materials, graphene}


\begin{document}
\maketitle

\begin{abstract}
 Very brief story.
\end{abstract}

\tableofcontents
%\listoffigures
%\listoftables

\chapter{Introduction}
\label{ch:introduction}
Tight-binding (TB) method is an approach to the calculation of energy band structure of various materials, where electrons are tightly bound to the atom to which they belongs. In TB model simple effective Hamiltonian is used in contrast to first-principle calculations. And an approximate set of wave functions based upon superposition of wave functions for isolated atoms located at atomic sites is used. All these make TB approach a great tool for investigating quite big systems and getting quantitative results.

TB method was developed by F. Bloch \cite{bloch} in 1928.  Bloch considered only the $s$ atomic orbital. In 1934 H. Jones, N. F. Mott and H. W. B. Skinner \cite{mott} considered wave functions constructed by solving a secular problem between $s$, $p_x$, $p_y$ and $p_z$ Bloch functions.

I developed a numerical package for Tight Binding calculations in Python programming language. With the help of this code one can investigate both regular bulk materials, which have translational symmetry in three dimensions and nanomaterials, which have reduced symmetry: $2$D -- nanosheets, surfaces, $1$D -- nanoribbons, nanorods, $0$D -- nanoparticles, molecules.

As an example of 3D system I provided results of band structure calculations for diamond. Furthermore I investigated graphene using different approaches and studied spin-orbit coupling effects. And finally I analyzed effects connected with periodicity of the system in graphene nanoribbons.

Package code and example inputs are placed in open repository \url{https://github.com/Lenka42/TightBinding}.
\chapter{Tight Binding method}
\label{ch:theory}
\section{Introduction to the tight-binding model}
Tight-binding method is developed to describe band structure of various materials, where electrons are localized at the atomic positions (tight-bonded to the atoms). In TB model simple effective hamiltonian is used in contrast to first-principle calculations. 

The most essential approximation in TB model is the so-called two-center approximation, where Hamiltonian is approximated by the atomic Hamiltonian centered on the atomic positions in the unit cell $\vec{R}$ and only two-centered integrals are encountered.

We construct the Bloch wave function of the crystal as a linear combination of the local Wannier functions, which are approximated by the eigenfunctions of the atomic Hamiltonian, the atomic orbitals $\phi_{\nu, s}(\vec{r} - \vec{t_i} - \vec{R})$, where $\vec{t_j}$ is the position of atom $i$ in the unit cell at $\vec{R}$ and $s$ is the spin of $\nu$th orbital. The Bloch wave function
\begin{equation} \label{eq:wave_fun}
\Psi_{\vec{k}, j ,\nu, s} = \frac{1}{\sqrt{N}} \sum_{\vec{R}} \me^{i\vec{k}\vec{R}} \phi_{\nu, s} (\vec{r} - \vec{t_j} - \vec{R})
\end{equation}
satisfies Bloch theorem.

Starting with Schr\"{o}dinger equation,
\begin{equation}
\hat{H}\Psi_{\vec{k}}(\vec{r}) = \epsilon_{\vec{k}} \Psi_{\vec{k}}(\vec{r})
\end{equation}
and expanding crystal wave function in the basis of the on-site Bloch wave functions
\begin{equation}
\Psi_{\vec{k}}(\vec{r}) = \sum_j c_{\vec{k}, j} \Psi_{\vec{k}, j}(\vec{r})
\end{equation}
basing on variational principle one can derive the secular equation,
\begin{equation} \label{eq:secular}
\sum_j[H_{i,j}(\vec{k}) - \epsilon_{\vec{k}} S_{i, j}(\vec{k})] c_{\vec{k}, j} = 0.
\end{equation}
Where two-center Hamiltonian and overlap matrix elements are defined by the integrals:
\begin{equation} \label{eq:h_matrix}
H_{i,j}(\vec{k}) = \frac{1}{N} \sum_{\vec{R}, \vec{R'}} \int d \vec{r} \phi_i^*(\vec{r} - \vec{R'}) \hat{H}(\vec{r} - \vec{R}) \phi_j(\vec{r} - \vec{R}),
\end{equation}
\begin{equation} \label{eq:s_matrix}
S_{i,j}(\vec{k}) = \frac{1}{N} \sum_{\vec{R}, \vec{R'}} \int d \vec{r} \phi_i^*(\vec{r} - \vec{R'}) \phi_j(\vec{r} - \vec{R}).
\end{equation}

Usually integration results are taken as parameters, which have to be fitted to reproduce certain solid's properties or the band structure calculated by first-principle approach. Matrix Elements in \ref{eq:h_matrix} and \ref{eq:s_matrix} where $\vec{R} = \vec{R'}$ are called on-site elements, and for $\vec{R} \neq \vec{R'}$ they are called hopping and overlap parameters respectively. 

In general case, atomic orbitals centered on different sites are not orthogonal and the corresponding overlap parameters have small but finite values.But often orthogonal basis of orbitals is used, so that overlap matrix becomes identity matrix and energies are eigenvalues of the Hamiltonian matrix. In the following sections I will discuss both approaches. 

The next approximation is to consider a finite as small as necessary number of orbitals per atom. The number of solutions of the secular equation in \ref{eq:secular} is equal to dimension of the Hamiltonian matrix,
\begin{equation}
dim = g \times \sum_j n_j
\end{equation}
Here $g$ is spin factor ($1$ or $2$), summation goes through all atoms in unit cell and $n_j$ is number of included orbitals for atom $j$.

In the final nearest neighbors approximation (NNA) only the nearest neighbors of a chosen atom are taken into account in the Hamiltonian and overlap matrix elements Eqs. (\ref{eq:h_matrix}) and (\ref{eq:s_matrix}).
\section{Spin-orbit interaction }

\chapter{Code discussion}
TB calculations, which results are presented in this work, are performed using package I wrote in Python programming language. Input should be also written in Python, but as far as Python is designed to be highly readable, even person who never programmed in Python before, can easily do this. And on the other hand this allows to simplify input data creation and further easy output rearrangement.

In this chapter I discuss the basic input example and what's going on when one runs it for calculating graphene band structure (Results of this input are presented in chapter \ref{subsec:graphene_result})

\begin{python}
from atom.model import Atom
from system.model import System
from plotter.plotter import Plotter

from numpy import array, sqrt, pi

d = 1.0
system = System([d / 2. * array([3., sqrt(3), 0.]),
                 - d / 2. * array([3, -sqrt(3), 0.])],
                mode='with_overlap', name='grapene_sp')
system.spin_multiplier = 2
\end{python}
 
At the beginning of the input file I do all the imports. First three are classes from the TightBinding package and the last is necessary functions from NumPy library. NumPy is a high-level package for efficient manipulating multi-dimensional arrays in Python. 

The next line defines a constant, which here represents length of $C-C$ bond in graphene.

At the beginning of logical part I define \verb!System! object. The first argument initializing the system object is list of lattice vectors. Length of this list may be from $0$ to $3$, depending on the periodicity of the system. Next argument is mode of calculation, it can be either \verb!"standard"!, \verb!"with_overlap"! when overlap matrix is not identity matrix or \verb!"with_vectors"! when one also wants eigenstates to be calculated. Name of the system is used to create directory with outputs.

Parameter \verb!spin_mutiplier! is set to $1$ by default. But if one wants to include SOC part to the Hamiltonian, one should assign value $2$ to this parameter.

\begin{python}
system.atoms = [Atom('C', array([d, 0., 0.])),
                Atom('C', array([2 * d, 0., 0.])), ]
\end{python}

Next I define list of atoms in the unit cell. Each atom has a name and a position.

\begin{python}
for i in xrange(len(system.atoms)):
    system.atoms[i].orbitals = ['s', 'px', 'py', 'pz', ]
\end{python}

Here I assign set of orbitals to each atom in a system. I do this in a loop, but one can also assign different sets of orbitals for every atom one by one.

\begin{python}
system.k_points = [array([0., 0., 0.]),
                   array([2 * pi / 3 / d, 2 * pi / 3 / sqrt(3) / d, 0]),
                   array([2 * pi / 3 / d, 0, 0]),
                   array([0., 0., 0.])]
system.make_k_mesh(100)
\end{python}

Thereafter one should define mesh of wave vector points. One can do this manually setting parameter \verb!system.k_mesh! or as above defining \verb!system.k_points! and calling method \verb!make_k_mesh!. In the snippet above I define a path of $\Gamma$, K, M and again $\Gamma$ points and call the function \verb!make_k_mesh! to create a mesh with approximately $100$ points following the path defined in the previous line.

\begin{python}
system.parameters = {
    'C': {
        'es': -8.7,
        'ep': 0.0,
    },
    'CC': {
        'Vsss': - 6.7,
        'Vsps': 5.5,
        'Vpps': 5.1,
        'Vppp': -3.1,
    }
}
system.s_parameters = {
    'C': {
        'es': 1.,
        'ep': 1.,
    },
    'CC': {
        'Vsss': 0.2,
        'Vsps': - 0.1,
        'Vpps': - 0.15,
        'Vppp': 0.12,
    }
}
\end{python}

Subsequently I define an intuitive set of parameters for my system. If parameter \verb!system.spin_multiplier! is $2$, parameter \verb!'lambda'! should be defined for each atom.

\begin{python}
system.just_do_main_magic()
plotter = Plotter(system.name)
plotter.plot_energy_bands_from_file()
\end{python}

Finally, I call the function \verb!just_do_main_magic! which does the main logic of the TB calculations. First this method calls helper function, which searches for nearest neighbors of each atom. Then for every wave vector in $system.k_mesh$ Hamiltonian (and overlap if needed) matrix is calculated. After eigenvalue problem is solved. I used functions from SciPy package for diagonalization of Hermitian matrices.

After I plot the electronic structure (this step is highly optional, as far as the results are stored in the files).

Also, after the main band structure calculation one may calculate density of states (DOS) or local density of states (LDOS) for any atom in the system or type of orbital.
\begin{python}
doser = DOSCalculator(system.dim, system.name, n)
doser.f()
\end{python}
Here first line is creation of helper object for DOS calculation, first argument is periodicity of the system, and \verb!n! is name of the system and the last is number of points in energy domain, for which one wants to calculate density of states. The next line is call of the function, which calculates DOS, writes results to the file and plots it.


To calculate LDOS:
\begin{python}
lst = system.find_indeces_for_ldos(orbital='d')
doser = LDOSCalculator(system.dim, system.name, n, localization_name, indeces_list=lst)
doser.f()
\end{python}

To be able co calculate LDOS one should first do the main TB calculations for the system in \verb!mode='with_vectors'!. First one should run helper function \verb!find_indeces_for_ldos! with keyword-argument \verb!orbital! or \verb!atom!. This function searches for all orbitals of certain type or on defined atom in the system. 

DOS and LDOS are calculated as described in \ref{ch:theory}. Interpolation of energy bands functions and integration are done with built-in SciPy methods.
\chapter{Diamond}
To start with I investigated regular $3D$ diamond crystal (fig. \ref{fig:diamond_lattice}). I considered here the case where we have only one set of $s$, $p_z$, $p_y$, and $p_z$ orbitals at each
atomic site (used parameters are placed at the table \ref{tab:diamond_params}). Calculated band structure is placed on the fig. \ref{fig:tb_diamond}. Agreement with experimental results \cite{diamond} (fig. \ref{fig:theory_diamond}) is quite good. However actual band structure of diamond has indirect band gap. And band structure calculated with TB is characterized by direct band gap at $\Gamma$ point.
\begin{figure}[h] 
 \begin{center}
  \includegraphics[width=0.3\linewidth]{img/diamond_crystall}
  \caption{Diamond crystal structure. \label{fig:diamond_lattice}}
 \end{center}
\end{figure}

\begin{table}[h]
 \begin{center}
  \begin{tabular}{|c|c|}
  \hline
    Parameter&Value, [eV]\\ \hline
    $\epsilon_s$ & $0.0$ \\ \hline
    $\epsilon_p$ & $7.4$ \\ \hline
    $V_{ss \sigma}$ & $-3.8$  \\ \hline
    $V_{sp \sigma}$ & $4.44$\\ \hline
    $V_{pp \sigma}$ & $-1.325$ \\ \hline
    $V_{pp \pi}$ &  $4.9$\\ \hline
  \end{tabular}
 \end{center}
  \caption{TB parameters for diamond. \label{tab:diamond_params}}
\end{table}

\begin{figure} 
  \includegraphics[width=\linewidth]{img/diamond_sp}
  \caption{Calculated band structure of diamond. \label{fig:tb_diamond}}
\end{figure}
\begin{figure} 
\begin{center}
  \includegraphics[width=0.5\linewidth]{img/diamond_exp_band_struct}
  \caption{Band structure of diamond (experiment). \label{fig:theory_diamond}}
\end{center}
\end{figure}  

\chapter{Molybdenum disulfide (monolayer)}
In my TB model I adopted a non-orthogonal basis set of $sp_3 d_5$ orbitals considering only the nearest neighbor interactions between following pairs of atoms: $Mo - S$, $Mo-Mo$, $S-S$.
\begin{figure}[ht]
\begin{center}
\begin{subfigure}{.5\textwidth}
  \centering
  \includegraphics[width=\linewidth]{img/mos2_all}
  \label{fig:mos2_all}
\end{subfigure}%
\begin{subfigure}{.5\textwidth}
  \centering
  \includegraphics[width=\linewidth]{img/mos2_zoom}
  \label{fig:mos2_zoom}
\end{subfigure}
  \caption{Calculated band structure of $MoS_2$. On the right -- zoom in the surrounding of Fermi level.}
\end{center}
\end{figure}  


\begin{figure}[ht]
\begin{center}
  \includegraphics[width=0.3\linewidth]{img/MoS2_lit_band_struc}
  \caption{Band structure of $MoS_2$ (literature)).}
\end{center}
\end{figure}  

\newpage
\section{Graphene}
\subsection{Very brief history of graphene}
Graphene is the most well-known 2D-material in the world.  Explosion of great interest to this material happened after Andre Geim and Konstantin Novoselov in 2010 received Nobel Prize in Physics "for groundbreaking experiments regarding the two-dimensional material graphene" \cite{geim}. 

But story of graphene started long years ago. Already in 1859 D.C. Brodie discovers the highly lamellar structure of thermally reduced graphite oxide \cite{brodie}. At the beginning of XXth century crystal structure of graphite was solved with diffraction method \cite{debije, bernal} and properties of graphite oxide paper was studied \cite{haenni}.

The theory of graphene was first studied by P. R. Wallace in 1947 as a first step to describing electronic properties of 3D graphite \cite{wallace}. The massless Dirac equation was first formulated for graphene by David P. DiVincenzo and Eugene J. Mele in 1984 \cite{divincenzo}. Semenoff emphasized the occurrence in a magnetic field of an electronic Landau level precisely at the Dirac point. This level is responsible for the anomalous integer quantum Hall effect.

At the ed of XXth century single layers of carbon atoms where epitaxially grown on other bulk materials\cite{epitaxial}. However strong interaction between graphene grown in such way and substrate significantly affects electronic structure and properties of graphene. In 2004 Andre Geim and Konstantin Novoselov published their work on the scotch tape method witch allowed to produce single layers of graphene weakly bounded to substrate on relatively large scale\cite{geim-science}. This paper initiated a global explosion in graphene research.

\subsection{Geometry of graphene}
Graphene is one-atom-sick sheet of carbon atoms arranged in hexagonal as shown on fig. \ref{fig:graphene_lattice}. The lattice vectors can be chosen as 
\begin{equation}
	\vec{a_1} = \frac{a}{2} (3, \sqrt{3}), \qquad \vec{a_2} = \frac{a}{2} (3, - \sqrt{3}),
\end{equation}
where $a \approx 1.42 \AA$ is $C-C$ bond length. There are two carbon atoms per unit cell, let's name them $A$ and $B$.The three nearest-neighbor vectors in real space are given by 
\begin{equation}
	\vec{\delta_1} = a (1, 0), \quad \vec{\delta_2} = \frac{a}{2} (-1, \sqrt{3}), \quad \vec{\delta_3} = \frac{a}{2} (-1, -\sqrt{3})
\end{equation}
and point from atom of one kind to another, e.g. from $A$ to $B$ and vice versa.
The reciprocal space lattice are given by
\begin{equation}
	\vec{b_1} = \frac{2 \pi}{3 a}(1, \sqrt{3}), \qquad \vec{b_2} = \frac{2 \pi}{3 a} (1, -\sqrt{3})
\end{equation}
Of particular interest for physics of graphene are two points in reciprocal space at the corners of Brillouin zone (BZ):
\begin{equation}
	K = (\frac{2 \pi}{3 a}, \frac{2 \pi}{3 \sqrt{3}}), \qquad K' = (\frac{2 \pi}{3 a}, -\frac{2 \pi}{3 \sqrt{3}})
\end{equation}

\begin{figure}[ht] \label{fig:graphene_lattice}
\begin{center}
  \includegraphics[width=0.55\linewidth]{img/graphene_lattice}
  \caption{Left: lattice structure of graphene. Right: corresponding Brillouin zone.}
\end{center}
\end{figure}

\subsection{Calculation of $\pi$ bands}
The easiest model one can use to describe electronic properties of graphene is TB model considering only $p_z$ orbitals for each carbon atom. All this orbitals are perpendicular to graphene plane and parallel to each other. They form $\pi$ bands. Band structure of graphene around the Fermi level is determined by the $\pi$ bands. 

Here I will take into account only interaction between nearest neighbors. One can also assume, that orbitals at different atoms are orthogonal to each other.

So to determine graphene band structure one should solve eigenvalue problem, where Hamiltonian matrix dimension is $2 \times 2$. The energy bands derived from this Hamiltonian have the form \cite{wallace}
\begin{equation}
	E(\vec{k}) = \pm V_{pp\pi}\sqrt{3 + 2 \cos\left(\sqrt{3} k_y a\right) + 4 \cos\left(\frac{\sqrt{3}}{2} k_y a\right) \cos\left(\frac{3}{2} k_x a\right)} 
\end{equation}

\begin{figure}[h] 
\begin{center}
  \includegraphics[width=0.55\linewidth]{img/graphene_pd_soc}
  \caption{Calculated Band structure of graphene ($p_3d_5$)).}
\end{center}
\end{figure}
\begin{figure}[h] 
\begin{center}
  \includegraphics[width=0.55\linewidth]{img/graphene_pd_soc3}
  \caption{Calculated Band structure of graphene ($p_zd_{xz}d_{yz}$)).}
\end{center}
\end{figure}

\chapter{Graphene nanoribbons}
The band structure of graphene nanoribbons is determined by the nature of their edges: armchair or zigzag. In figure \ref{fig:edges} there is a piece of honeycomb lattice, which have zigzag edges along $x$ direction and armchair edge along the $y$ direction. 
\begin{figure}[h] 
\begin{center}
  \includegraphics[width=0.3\linewidth]{img/nanoribbbon_edges}
  \caption{A piece of honeycomb lattice with both zigzag and armchair edges. \label{fig:edges}}
\end{center}
\end{figure}


\section{Armchair nanoribbons}
Armchair nanoribbons can have either metallic or semiconducting character, depending on their width. In the TB calculations metallic armchair nanoribbons have energy bands crossing with linear dispersion. In fig. \ref{fig:ac_ribbons}
I show calculated band structures for a set of ribbons with different widths. Nanoribbons of widths $n= 11$ (fig. \ref{fig:ac11}) and $n=32$ (fig. \ref{fig:ac32}) have zero energy gap. However, DFT calculations show that armchair nanoribbons have non-zero energy gap scaling with the inverse of the ribbon width \cite{han}.
\begin{figure}[hb!]
\centering
\begin{subfigure}{.5\textwidth}
  \centering
  \includegraphics[width=\linewidth]{img/ac_ribbon_10}
  \caption{$n=10$}
  \label{fig:ac10}
\end{subfigure}%
\begin{subfigure}{.5\textwidth}
  \centering
  \includegraphics[width=\linewidth]{img/ac_ribbon_11}
  \caption{$n=11$}
  \label{fig:ac11}
\end{subfigure}
\begin{subfigure}{.5\textwidth}
  \centering
  \includegraphics[width=\linewidth]{img/ac_ribbon_32}
  \caption{$n=32$}
  \label{fig:ac32}
\end{subfigure}%
\begin{subfigure}{.5\textwidth}
  \centering
  \includegraphics[width=\linewidth]{img/ac_ribbon_33}
  \caption{$n=33$}
  \label{fig:ac33}
\end{subfigure}
\begin{subfigure}{.5\textwidth}
  \centering
  \includegraphics[width=\linewidth]{img/ac_ribbon_100}
  \caption{$n=100$}
  \label{fig:ac100}
\end{subfigure}
\caption{Calculated band structure of armchair nanoribbons of various widths $n$. \label{fig:ac_ribbons}}
\end{figure}

\section{Zigzag nanoribbons}
In TB model calculations zigzag nanoribbons are always metallic and have a zero-energy band. This dispersionless zero-energy mode is the surface states confined near the edge of the graphene ribbon. However in the experiments graphene nanoribbons are highly disordered at the edges \cite{areshkin} and often passivated by hydrogen \cite{barone}. All these strongly effect the properties of edge states. On fig. \ref{fig:zz_ribbons} I show calculated spectrum of zigzag graphene nanoribbons of various widths.
\begin{figure}[hb!]
\centering
\begin{subfigure}{.5\textwidth}
  \centering
  \includegraphics[width=\linewidth]{img/zz_ribbon_2}
  \caption{$n=2$}
  \label{fig:zz2}
\end{subfigure}%
\begin{subfigure}{.5\textwidth}
  \centering
  \includegraphics[width=\linewidth]{img/zz_ribbon_6}
  \caption{$n=6$}
  \label{fig:zz12}
\end{subfigure}
\begin{subfigure}{.5\textwidth}
  \centering
  \includegraphics[width=\linewidth]{img/zz_ribbon_20}
  \caption{$n=20$}
  \label{fig:zz6}
\end{subfigure}%
\begin{subfigure}{.5\textwidth}
  \centering
  \includegraphics[width=\linewidth]{img/zz_ribbon_64}
  \caption{$n=64$}
  \label{fig:zz64}
\end{subfigure}
\begin{subfigure}{.5\textwidth}
  \centering
  \includegraphics[width=\linewidth]{img/zz_ribbon_200}
  \caption{$n=200$}
  \label{fig:zz200}
\end{subfigure}
\caption{Calculated band structure of zigzag nanoribbons of various widths $n$.\label{fig:zz_ribbons}}
\end{figure}

\chapter{Summary}

As a result of this work open-source Python package for Tight Binding calculation appeared. 

I investigated diamond's energy band structure and showed that results agree with the results of TB calculations provided in the literature. However according to the calculated band structure diamond has direct band gap in $\Gamma$ point, whereas real diamond crystal is characterized by indirect band gap.

I studied graphene starting with the simplest model. Next I introduced expanded set of orbitals and nontrivial overlap matrix. And finally I showed, that intrinsic band gap due to spin orbit coupling appears in graphene when $d$ orbitals are taken into account.

At the last part of my work I investigated graphene nanoribbons and showed that band structure and their character depend on the nanoribbons' edges and their width. Also I studied density of states of the system with reduced periodicity. It appears, that size of sample crystal of the order of $100$ atoms is enough to describe crystal using wave-vector formalism.

\begin{thebibliography}{99}
\addcontentsline{toc}{chapter}{Bibliografia}

\bibitem{bloch} F. Bloch, \textit{Über die Quantenmechanik der Elektronen in Kristallgittern}, Z. Phys. (1928), vol. 52: 555–600
\bibitem{mott} H. Jones, N. F. Mott,  H. W. B. Skinner, \textit{A Theory of the Form of the X-Ray Emission Bands of Metals}, Phys. Rev. 45, 379 (1934)
\bibitem{slatter} J. C. Slater, G. F. Koster, \textit{Simplified LCAO Method for the Periodic Potential Problem}, Phys. Rev. 94, 1498 (1954)
\bibitem{kittel} C. Kittel, \textit{Introduction to Solid State Physics, 7th ed.}, Wiley, (1996) 
\end{thebibliography}


\end{document}
